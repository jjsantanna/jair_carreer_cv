\documentclass[]{friggeri-cv} % Add 'print' as an option into the square bracket to remove colors from this template for printing

\usepackage{amssymb, amsmath} 

\addbibresource{mylibrary.bib}

\begin{document}

\header{José Jair }{Santanna}{Enthusiastic, Researcher, Lecturer, Network Security Specialist \& (Big) Data Analyst} 

\begin{aside} % In the aside, each new line forces a line break
\section{Contact}
Drienerlolaan 5,
Zilverling 5110,
7522 NB Enschede,
The Netherlands.
~
+31 (6) 42330703
~
\href{http://jairsantanna.com}{\textbf{http://jairsantanna.com}}
\href{mailto:j.j.santanna@utwente.nl}{j.j.santanna@utwente.nl}
\section{Languages}
portuguese native 
english fluency
spanish comprehension
dutch comprehension
\section{Programming}
Bash, Python, C\#,SQL
PIG, Spark \& Impala
\end{aside}

\section{Professional Objective}
%What are your current responsibilities?Why did you come to the UT? Why is this the right host for your research?
I am currently a PhD candidate at University of Twente (ending in 2017) in a research group called Design and Analysis of Communication Systems (DACS). My research is founded by the European commission via a project focus in network of excellence, called Flamingo. In addition, my research is funded by one of the main Dutch national Organisation for Scientific Research (NWO) in a project called Distributed Denial of Service Defence (D3).

I am interested in a research \& development position with focus on network security and the use of big data technologies to improve network data analysis. A position where I can transfer my enthusiasm, knowledge, skills and experience into meaningful innovation to improve the security in computer networks, specially in the Internet. 

\section{Education}

\begin{entrylist}

\entry
{2013--Now}
{Doctor {\normalfont in Computer Science}}
{University of Twente, The Netherlands}
{\emph{``The DDoS as a \$ervice Phenomenon''.} \\
Supervisor: Prof. dr. ir. Aiko Pras 
}	

\entry
{2010--2012}
{Master {\normalfont in Computer Science}}
{Federal University of Rio Grande do Sul, Brazil}
{\emph{``A BPM-based Solution for Inter-domain Circuit Management''.} The outcome of this work generated a product still in use by the Brazilian National Education and Research Network (RNP) (Link: \href{http://meican.cipo.rnp.br/}{http://meican.cipo.rnp.br/}).\\
Supervisor: Prof. dr. Lisandro Zambenedetti Granville}

\entry
{2005--2010}
{Bachelor {\normalfont in Computer Engineer}}
{Federal University of Par\'a, Brazil}
{\emph{``A Comparative Evaluation of DCCP Protocol in Mesh Networks.''} \\ 
Supervisor: Dr. Kelvin Lopes Dias
%This dissertation ...
}

\end{entrylist}

\section{Experience}

\subsection{Technical Adviser}
\begin{entrylist}
\entry
{2016}
{To the Yokohama National University}
{Japan}
{For the production of a report on ``Reflection \& Amplification DDoS attacks'' requested by the \emph{Japanese Ministry of Internal Affairs and Communications}.}

\entry
{2016}
{To the Dutch National Cyber Security Center and the Scientific Research \& Documentation Center}
{The Netherlands}
{For the production of a report on ``Cyber metrics 2016: DDoS and Malware'' requested by the \emph{Dutch Ministry of Security and Justice}.}
\end{entrylist}

\subsection{Assistant Professor}
\begin{entrylist}
\entry
{2012--2013}
{University of Santa Cruz do Sul (UNISC)}
{Santa Cruz do Sul, Brazil}
{Tasks \& Responsibilities: Supervise students and teach the following courses: Computer Networks, Computer Network Management, and Digital Systems.}
\end{entrylist}

\newpage
\subsection{Guest Lecturer}
\begin{entrylist}
\entry
{2015 \& 2016}
{Product Design to Online Business (course)}
{University of Twente, The Netherlands}
{Audience: bachelor students of industrial engineering and business information technology.}

\entry
{2015}
{Network Security (course)}
{University of Twente, The Netherlands}
{Audience: master and bachelor students of computer science, electrical engineer, and telematics from the University of Twente, 3TU cyber security, and members of ICT labs. }

\entry
{2015}
{Cybercrime \& Cybersecurity (minor)}
{University of Twente, The Netherlands}
{Audience: bachelor students from University of Twente, European Research Center for Information (ERCIS), Westfälische Wilhelms - Univerität Münster, Universität Innsbruck, and University of Leicester. }
\end{entrylist}

\subsection{Student Supervisor}
Among 10 supervised students: 5 decided to follow towards the next academic degree[$^\uparrow$], PhD or MSc, 2 received the MSc degree cum laude[$^\Diamond$], 2 won the best paper award in international conferences[$^\star$], and 1 won the second place of a renowned Dutch national prize (KIVI/Defensie \& Veiligheid) for her thesis[$\star$].

\begin{entrylist}
\vspace{-0.3cm}
\entry
{[M.Sc. Degree]}
{Justyna Chromik[$^\uparrow$ $^\Diamond$ $^\star$ $\star$] , Wouter de Vries[$^\uparrow$ $^\star$], Joey de Vries[$^\Diamond$], Mark Wierbosch, and Jarmo van Lenthe.}
{}

\vspace{-0.3cm}
\entry
{[B.Sc. Degree]}
{Roeland Krak[$^\uparrow$], Max Kerkers[$^\uparrow$], Dirk Maan[$^\uparrow$], Jochum Börger, and Guilherme Dressler.}
{}

%[MINOR in Cyber Attacks and Cyber Security ] Karel Kroonen, Michiel Hamilton, Rick Clement, Peter Schroten, Yuri van Midden, Benedict Steinhoff, Kilian Müller, Stephanie Massa, Sören Schleibaum
%[DESIGN Project] Tom Velthausz, Henk Mulder, Thomas Reesink, Jordy Michorius,Dennis Eijkel;
%[Crossing Border Module]Julik Keijer

\end{entrylist}

\subsection{International Conference Reviewer}
The first four listed conferences are among the top well ranked related to the Committee on Network Operation and Management (CNOM), which is a renowned technical committee part of the IEEE Communications Society. For each conference we present the acronym, the full name, the acceptance rate of paper submitted divided by published* (according \href{http://www.cs.ucsb .edu/~almeroth/conf/stats}{\texttt{www.cs.ucsb .edu/$\textasciitilde$almeroth}}), and the h-index* (according Google scholar).

\begin{entrylist}

\vspace{-0.3cm}
\entry
{CNSM} %(2013-- 2015)
{International Conference on Network and Service Management}
{17.6\% | 18}

\vspace{-0.3cm}
\entry
{NETWORKING} %(2014--2016)
{IFIP Networking}
{23.8\% | 22 }

\vspace{-0.3cm}
\entry
{IM} %(2015)
{IFIP/IEEE Integrated Management}
{27.2\% | 18}

\vspace{-0.3cm}
\entry
{NOMS} %(2015)
{IEEE/IFIP Network Operations and Management Symposium }
{28.9\% | 22}

\vspace{-0.3cm}
\entry
{AIMS} %(2015)
{Autonomous Infrastructure, Management and Security}
{31.8\% | 9}

\vspace{-0.3cm}
\entry
{SCC} %(2013)
{IEEE Symposium on Computers and Communications}
{46.4\% | 20}

\vspace{-0.3cm}
\entry
{APNOMS} %(2013 \& 2014))
{Asia-Pacific Network Operations and Management Symposium}
{ -- | 10}

\end{entrylist}

\let\thefootnote\relax\footnotetext{*last checked: April 7,2016}


\newpage
\subsection{Software Developer}
\begin{entrylist}

\entry
{2015--Now}
{Distributed Denial of Service Attacks Database (DDoSDB)}
{\\Link: \href{http://ddosdb.net}{http://www.ddosdb.net} }
{An initiative to develop the largest open database with DDoS attacks.\\
Languages: Python, Bash, Pig Latin \& Apache Impala
}

\entry
{2010--2011}
{{Management Environment of Inter-domain Circuits for Advanced Networks (MEICAN)}}
{Link: \href{http://meican.cipo.rnp.br/}{http://meican.cipo.rnp.br/}}
{Development of an Executable Network Management Policies Editor sponsored by the Brazilian National Education and Research Network (RNP).    \\
Language: WS-BPEL
}

\entry
{2009}
{Support for Planning Sustainable Electrical Power Generation via Vegetable Oil.}
{}
{Desktop application part of a Ph.D. thesis [\href{https://sites.google.com/site/ceamazonufpa/TeseAnaRosaDuarte.pdf}{link}] sponsored by a subsidiary (Eletronorte) of the major Brazilian power utility (Eletrobrás).\\
Language: C\#.Net \& mySQL
}
\end{entrylist}

\section{Presentations}
%most reviewers will not take time to check this website, so you have to give a bit more information. For instance how many at conferences, which conferences, how many invited, where? etc. 
\emph{More than 30 presentations in the last 3 years (available on my  \href{http://www.slideshare.net/jjcsantanna}{slideshare.net/jjcsantanna}), among those I highlight:}

\begin{entrylist}

\vspace{-0.3cm}
\entry
{2015}
{Oral Presentation}
{at Logius (part of the Dutch Ministry of the Interior and Kingdom Relations).}

\vspace{-0.3cm}
\entry
{2015}
{Oral Presentation}
{to the Dutch National Management Organization of Internet Providers (NBIP).}

\vspace{-0.3cm}
\entry
{2015}
{Oral Presentation}
{to the Dutch National Cyber Security Centrum (NCSC).}

\vspace{-0.3cm}
\entry
{2013 \& 2015}
{Oral Presentation}
{at the 31$^{th}$ and 36$^{th}$ IRTF Network Management Research Group (NMRG).}

\end{entrylist}

\section{Awards}

\begin{entrylist}
\vspace{-0.3cm}
\entry
{2015}
{Best Ph.D. 'Elevator Pitch' (1$^{st}$ Place)}
{CTIT Symposium, University of Twente}

\vspace{-0.3cm}
\entry
{2015}
{Best Poster (3$^{rd}$ Place)}
{CTIT Symposium, University of Twente}

\vspace{-0.3cm}
\entry
{2015}
{Best Paper Award}
{Autonomous Infrastructure, Management and Security Conference}

\vspace{-0.3cm}
\entry
{2015}
{Top Papers Nomination}
{Brazilian Symposium on Computer Networks and Distributed Systems}

\vspace{-0.3cm}
\entry
{2010}
{Student Award (3$^{rd}$ Place)}
{Computer Engineer, Federal University of Pará, Brazil}

\end{entrylist}


\let\thefootnote\relax\footnotetext{CV last update: \today}

\newpage
\section{Scientific Appendix}

%\subsection{Description of Ph.D. Research}
\subsection{Publications}
%what are the top conferences  / journals in your field? 
\nocite{*}
\printbibliography[title={\emptyset}]

\end{document}