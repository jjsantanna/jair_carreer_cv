%ATTENTION: Compile with XeLatex and Biber!!!

\documentclass[print]{styles/friggeri-cv-mac} % Add 'print' as an option into the square bracket to remove colors from this template for printing
%\documentclass[print]{styles/friggeri-cv-linux} % Add 'print' as an option into the square bracket to remove colors from this template for printing

\usepackage{soul}
\addbibresource{mylibrary.bib}

\begin{document}
\header{Jos\'e Jair Cardoso de }{Santanna}
{Enthusiastic Researcher, Internet Security Specialist and (Big) Data Analyst} 

\begin{aside} 
 \section{Contact}
 Hulsmaatstraat 73
 7523WB Enschede
 The Netherlands
%University of Twente
%DACS research group
%Drienerlolaan 5 
%7522 NB Enschede
%The Netherlands
~
\href{mailto:jairsantanna@gmail.com}{jairsantanna@gmail.com}
+31 6 42330703
~
\section{Links}\hspace{-1cm}
\hspace{-0.5cm}
%\href{http://www.jairsantanna.com}{jairsantanna.com}\includegraphics[scale=0.4]{img/jairsantanna.png}
\href{https://www.linkedin.com/in/jjcsantanna}{/jjcsantanna}\includegraphics[scale=0.3]{img/linkedin.png}
\href{https://github.com/jjsantanna}{/jjsantanna\includegraphics[scale=0.3]{img/github.png}}
\href{http://www.slideshare.net/jjcsantanna}{/jjcsantanna}\includegraphics[scale=0.3]{img/slideshare.png}
\href{https://scholar.google.com/citations?user=TxcQNxUAAAAJ}{/TxcQNxUAAAAJ}\includegraphics[scale=0.3]{img/googlescholar.png}
\href{http://orcid.org/0000-0002-8361-6729}{/0...2-8361-6729}\includegraphics[scale=0.3]{img/orcid.png}
\href{http://www.researcherid.com/rid/K-2169-2016}{/K-2169-2016}\includegraphics[scale=0.3]{img/researchid.png}
%\href{https://www.facebook.com/jjcsantanna}{/jjcsantanna}\includegraphics[scale=0.3]{img/facebook.png}
%~
%\section{Programming}
%	Bash, Python, C\#,SQL
%	PIG, Spark/Scala \& Impala
~
\section{Languages}
Dutch\includegraphics[scale=0.40]{img/2stars.png}
English\includegraphics[scale=0.40]{img/5stars.png}
Portuguese\includegraphics[scale=0.40]{img/5stars.png}
\end{aside}

\section{Status and Professional Objective}\vspace{-10pt}
\noindent\setlength\parindent{12pt}\textit{
	I am currently assistant professor at University of Twente, I provide technical support to the Team High Tech Crime Unit Police on digital forensics, and I am the product manager of the Dutch DDoS Clearing House (a coalision of 25 players from industry (ISPs, xSPs, IXPs, banks, not-for-profit DPS) and gov’t (ministries and agencies)).
	%
	A Ph.D. degree means: capacity to think analytically and abstractly to solve complex problems. 
	%
	In the period between 2013 and 2017, I was a Ph.D. candidate at the same university. A number of solutions proposed in my Ph.D. thesis were deployed by network operators worldwide and some methodologies will be used by the Dutch High Tech Crime Unit. During my Ph.D., I also acted as technical advisor to the Dutch National Cyber Security Center and to the Japanese Ministry of Internal Affairs and Communications, both on the topic of Cyber Security and DDoS attacks. 
	%
	I presented my work in more than 30 countries. In 2018, I won a national competition as "the best science communicator of The Netherlands” and I was elected as one of the best lectures of the computer science faculty in the University of Twente.
%My professional objective is to become an internationally recognized researcher and teacher in DDoS attack, Internet security and big data analysis. I aim to transfer my enthusiasm, knowledge, skills and experience into meaningful innovation to improve the security in all types of computer networks, especially the Internet.
	My professional objective is to improve the security in all types of computer networks, especially the Internet by transferring my enthusiasm, knowledge, skills and experience into high impact solutions. 
}

%} \vspace{-5pt}\setlength\parindent{12pt}\textit{


%==============================================================================
%==============================================================================
%==============================================================================
\setlength\parindent{0pt}
\section{Education}\vspace{-5pt}

\begin{entrylist}

\entry
{2013--2017}
{Doctor {\normalfont in Computer Science}}
{University of Twente, The Netherlands}
{\emph{\href{https://research.utwente.nl/files/18494043/jjsantanna_thesis.pdf}{``DDoS-as-a-Service--Investigating Booter Websites''.}} \\
Supervisor: Prof. Dr. Ir. Aiko Pras\\	
\textit{This work has being used by both the security community and by
network operators worldwide. One of the outcomes that stands out is publicly available at \href{http://booterblacklist.com}{`booterblacklist.com'} and  already downloaded by more than three thousand users worldwide.}}

\entry
{2010--2012}
{Master {\normalfont in Computer Science}}
{Federal University of Rio Grande do Sul, Brazil}
{\emph{``A BPM-based Solution for Inter-domain Circuit Management''.}\\
Supervisor: Prof. Dr. Lisandro Zambenedetti Granville\\
\textit{The outcome of this work is still in production used by the Brazilian
National Education and Research Network and available at
\href{http://meican.cipo.rnp.br/}{`meican.cipo.rnp.br'}.}}

\entry
{2005--2010}
{Bachelor {\normalfont in Computer Engineer}}
{Federal University of Par\'a, Brazil}
{\emph{``A Comparative Evaluation of DCCP Protocol in Mesh Networks.''} \\ 
Supervisor: Dr. Kelvin Lopes Dias
}

\end{entrylist}

%==============================================================================
%==============================================================================
%==============================================================================
\section{Experience}\vspace{-5pt}
%==============================================================================
\subsection{Product Owner}\vspace{-5pt}
\begin{entrylist}
\entry {2018--current}	{The DDoS Clearing House}
{\\Funded by: a coalition of 25 players from industry (ISPs, xSPs, IXPs, banks, not-for-profit DPS) and gov’t (ministries and agencies)}
{}
\end{entrylist}
%==============================================================================
\subsection{Project Leader}\vspace{-5pt}
\begin{entrylist}
\entry{2017--current}{Collecting, Transforming, Applying, and Disseminating DDoS Attack Knowledge }
{\\€200k Funded by: SIDN fonds}
{}

\entry{2018--2019}{Online Discoverability and Vulnerabilities of ICS/SCADA Devices in the Netherlands}
{\\€80k Funded by: Wetenschappelijk Onderzoek- en Documentatiecentrum (WODC)}
{}
\end{entrylist}
%==============================================================================
\subsection{Task Leader}\vspace{-5pt}
\begin{entrylist}
\entry	{2019--2023}{Cybersecurity Competence for Research and Innovation (CONCORDIA)}
{\\€16M Funded by: the European Framework Programme for Research and Innovation}
{Responsible to lead 42 partners on the investigation of network-centric security solutions for Europe.}	
\end{entrylist}
%==============================================================================
\subsection{Technical Advisor}\vspace{-5pt}
\begin{entrylist}
\entry
{2019--2019}{To the \textit{Nationale Beheersorganisatie Internet Providers} (NBIP) and the Stichting Internet Domeinregistratie Nederland (SIDN)}
{The Netherlands}
{Report on: ``The impact of DDoS attacks on Dutch enterprises.''}
	
\entry
{2016--2016}
{To the Yokohama National University}
{Japan}
{For the production of a report on ``Reflection \& Amplification DDoS attacks''
requested by the \emph{Japanese Ministry of Internal Affairs and
Communications}.}

\entry
{2016--2016}
{To the Dutch National Cyber Security Center and the Scientific Research \& Documentation Center}
{The Netherlands}
{For the production of a report on ``Cyber metrics 2016: DDoS and Malware'' requested by the \emph{Dutch Ministry of Security and Justice}.} 

\end{entrylist}

\subsection{Lecturer}\vspace{-5pt}
\begin{entrylist}
\entry
{2012--2013}
{University of Santa Cruz do Sul (UNISC)}
{Santa Cruz do Sul, Brazil}
{Tasks \& Responsibilities: Supervise students and teach (integrally) the courses: \\
	\textbf{$\bullet$~Digital Systems;\\$\bullet$~Computer Networks; and\\$\bullet$~Network Management}.} 
\end{entrylist}

%==============================================================================
%==============================================================================
%==============================================================================
\newpage
\subsection{Guest Lecturer}\vspace{-5pt}

%\textit{Most of my slides are available at
%\href{http://www.slideshare.net/jjcsantanna}{slideshare.net/jjcsantanna}; some
%interactive exercises are available at
%\href{https://github.com/jjsantanna/lectures_hands_on}{github.com/jjsantanna/lectures\_hands\_on};
%and, most important, there is some voluntary judgment and testimonies of students/attendees available at
%\href{http://bit.ly/presentation_evaluation}{bit.ly/presentation\_evaluation}.}

\begin{entrylist}
\entry
{2016 \& 2017}
{Network Management}
{University of Twente, The Netherlands}
{Lectures: (1) `` SNMPv1, v2, v3 and Beyond'' and (2) ``Network management based
	on WebServices''\\
Audience: master students of computer science, electrical engineer, and
telematics from the University of Twente. }

\entry
{2015 \& 2016}
{Product Design to Online Business}
{University of Twente, The Netherlands}
{Lectures: (1) ``Booters: the DDoS-as-a-Service phenomenon'' and (2) ``Product
Design to Online Business''\\ Audience: bachelor students of industrial
engineering and business information technology.}

\entry
{2015}
{Network Security}
{University of Twente, The Netherlands}
{Lecture: ``DDoS attacks into the Matrix''\\
Audience: master and bachelor students of computer science, electrical engineer,
and telematics from the University of Twente, 3TU cyber security, and members of ICT labs. }

\entry
{2015}
{Cybercrime \& Cybersecurity (minor)}
{University of Twente, The Netherlands}
{Lecture: ``DDoS attacks - Back to the future''\\ 
Audience: bachelor students from University of Twente, European Research Center
for Information (ERCIS), Westf\"alische Wilhelms �Universit\"at M\"unster, Universit\"at Innsbruck, and University of Leicester. } \end{entrylist}

%==============================================================================
%==============================================================================
%==============================================================================
\subsection{Student Supervisor}\vspace{-5pt}

\textit{Among 10 supervised students: 6 had their research published in
international academic conferences, 5 decided to follow towards the next
academic degree [$^*$] (Ph.D. or M.Sc.), 2 received the M.Sc. degree
\textit{cum laude} [$^+$], 1 won the best paper award in an international
conference [$^\#$], 1 won the second place of a renowned Dutch national
prize (KIVI/Defensie \& Veiligheid) for her thesis [$!$], and 2 are working as Digital Expert for the Dutch National High Tech Crime Unit [$§$].}

\begin{entrylist}
\vspace{-0.3cm}
\entry
{[\textbf{M.Sc. Degree}]}
{\textnormal{Kareem Fouda, Justyna Chromik[$^*$,$^+$,$!$] , Wouter de Vries[$^*$,$^\#$], Joey de Vries[$^+$], Mark Wierbosch, and Jarmo van Lenthe[$§$].}}
{}

\vspace{-0.3cm}
\entry
{[\textbf{B.Sc. Degree}]}
{\textnormal{Calvin Hendriks, Roeland Krak[$^*$], Max Kerkers[$^*$], Dirk Maan[$^*§$], Jochum Börger, and Guilherme Dressler.}} 
{}

\end{entrylist}

%\vspace{-0.3cm}
%\textit{Besides those students I also supervised: two groups of students from University of Twente and University of Münster in the ``Cyber Attacks and Cyber Security''
%(minor); a group of 5 students in a ``Design Project''; and 1 student in the
%``Crossing borders program''.}

%[MINOR in Cyber Attacks and Cyber Security ] Karel Kroonen, Michiel Hamilton,
%Rick Clement, Peter Schroten, Yuri van Midden, Benedict Steinhoff, Kilian
%Müller, Stephanie Massa, Sören Schleibaum [DESIGN Project] Tom Velthausz, Henk
%Mulder, Thomas Reesink, Jordy Michorius,Dennis Eijkel; [Crossing Border
%Module]Julik Keijer

%==============================================================================
%==============================================================================
%==============================================================================
\subsection{Journal and Conference Reviewer}\vspace{-5pt}
%\textit{Among more than 20 venues acting as a
%reviewer I highlight bellow only journals (3) and the (3) top well ranked conferences by the
%(IEEE) Committee on Network Operation and Management (CNOM). While for journal
%I added the `impact factor' and the `h-index' metrics, for conferences the
%`acceptance rate' and, also, the `h-index', respectively.}

\begin{entrylist}
\vspace{-0.3cm}
\entry {TNSM} {Transactions on Network and Service Management} {3.134 | 25}
	
\vspace{-0.3cm}
\entry {COMCOM} {Computer Communications} {2.099 | 49}

\vspace{-0.3cm}
\entry {JNSM} {Journal of Network and Systems Management} {0.796 | 15}

\vspace{-0.3cm}
\entry {IJNM} {International Journal of Network Management} {0.681 | 10}

\vspace{-0.3cm}
\entry{CNSM} {International Conference on Network and Service Management} {17.6\% | 18}

\vspace{-0.3cm}
\entry {NETWORKING} {IFIP Networking} {23.8\% | 22 }

\vspace{-0.3cm}
\entry {IFIP IM} {IFIP/IEEE Integrated Management} {27.2\% | 18}

\vspace{-0.3cm}
\entry {NOMS} {IEEE/IFIP Network Operations and Management Symposium } {28.9\% | 22}
\end{entrylist}


%%%%FROM JEMS
%AIMS 	(2015)
%APNOMS (2013,2014)
%ERRC 	(2012,2013,2015,2016)
%IM 	(2015)	
%NOMS 	(2014, 2016)
%SBRC 	(2012)
%SBrT 	(2011,2012)
%SBSeg 	(2012) 
%WSL 	(2010)

%%%%FROM EDAS
%CCSNA	(2016)
%CNSM	(2016,2015,2014,2013)
%Networking (2016,2015,2014)
%EUNICE	(2013)
%ISCC	(2011)

%==============================================================================
%==============================================================================
%==============================================================================
\newpage
\subsection{Conference Organizer}\vspace{-5pt}

\begin{entrylist}
\entry
{2016} 
{Shadow Technical Program Committee (TPC)}
{CNSM}
{\textit{IFIP/IEEE International Conference on Network and Service Management is
the premier conference in the area of network and service management. A shadow
TPC is an experimental initiative for PhDs become better TPCs in the future. My
reviews have the same weight as the actual TPC reviews.}}

\entry
{2013-2016} 
{Technical Program Committee (TPC)}
{ERRC}
{\textit{Escola Regional de Redes de Computadores (pt\_br), a conference for
undergrad students from the south of Brazil. To motivate students to }}

\entry
{2012} 
{Local Organizing Committee}
{LatinCloud}
{\textit{IEEE Latin American Conference on Cloud Computing and Communications.}}

\entry
{2010}
{Local Organizing Committee}
{SBRC}
{\textit{The Brazilian Symposium on Computer Networks and Distributed Systems.}}

\end{entrylist}

%==============================================================================
%==============================================================================
%==============================================================================
%\subsection{Project Activities}\vspace{-5pt}
%Management of Hybrid Networks (HyMan)

%Desenvolvimento de Tecnologia para Redes Metropolitanas Sem Fio voltada para Serviços de Cidades Inteligentes, Descrição: O projeto tem como objetivo desenvolver uma solução integrada para os componentes de escalonamento e controle de admissão para dispositivos de 4 geração utilizando a tecnologia WiMAX.. , Situação: Em andamento; Natureza: Pesquisa. , Alunos envolvidos: Graduação: (2) . , Integrantes: José Jair Cardoso de Santanna - Integrante / Cristiano Bonato Both - Coordenador.

%2007 - 2009
%Arcabouço para Acesso ao Conhecimento à População da Região Amazônica, 
%Descrição: Abordamos aspectos de pesquisa nas áreas de infra-estrutura de comunicações, implementação e avaliação de interfaces amigáveis para acesso universal aos cidadãos, considerando as oportunidades de conectividade existentes e peculiaridades da região amazônica. O estudo e solução propostos neste projeto consideram hardware de comunicação de baixo custo, além de software livre e aberto para a composição de um sistema de acesso flexível e escalável.. , Situação: Concluído; Natureza: Pesquisa. , Alunos envolvidos: Graduação: (3) / Mestrado acadêmico: (7) / Doutorado: (2) . , Integrantes: José Jair Cardoso de Santanna - Integrante / Kelvin Lopes Dias - Coordenador., Financiador(es): Conselho Nacional de Desenvolvimento Científico e Tecnológico - Auxílio financeiro.


%==============================================================================
%==============================================================================
%==============================================================================
%\subsection{Software Developer}\vspace{-5pt}
%\begin{entrylist}
%
%\entry
%{2015--Now}
%{Distributed Denial of Service Attacks Database (DDoSDB)}
%{\\Link: \href{http://ddosdb.org}{http://www.ddosdb.org} }
%{An initiative to develop the largest open database with DDoS attacks.\\
%Languages: Python, Bash, Pig Latin \& Apache Impala
%}
%
%\entry
%{2010--2011}
%{{Management Environment of Inter-domain Circuits for Advanced Networks (MEICAN)}}
%{Link: \href{http://meican.cipo.rnp.br/}{http://meican.cipo.rnp.br/}}
%{Development of an Executable Network Management Policies Editor sponsored by the Brazilian National Education and Research Network (RNP).    \\
%Language: WS-BPEL
%}

%\entry
%{2009}
%{Support for Planning Sustainable Electrical Power Generation via Vegetable Oil.}
%{}
%{Desktop application part of a Ph.D. thesis [\href{https://sites.google.com/site/ceamazonufpa/TeseAnaRosaDuarte.pdf}{link}] sponsored by a subsidiary (Eletronorte) of the major Brazilian power utility (Eletrobrás).\\
%Language: C\#.Net \& mySQL
%}
%\end{entrylist}



%==============================================================================
%==============================================================================
%==============================================================================
\section{Presentations}\vspace{-5pt}

%\textit{In the last 3 years I gave more than 30 presentations, which is almost
%one per month. Most of the presentations are available at
%\href{http://www.slideshare.net/jjcsantanna}{`slideshare.net/jjcsantanna'}.
%Among invited presentations, presentations that required abstract and 
%presentations approved after a very strict peer-reviewing process (conferences)
%I highlight the following.}

\begin{entrylist}
	
\vspace{-0.3cm}
\entry {\textbf{Abstract}}{\textnormal{ONE conference (main stage) *together with Dutch Team High Tech Crime Unit Police}} {2018}	

\vspace{-0.3cm}
\entry
{\textbf{Invited}}
{\textnormal{To Logius (part of the Dutch Ministry of Interior and Kingdom Relations)}}
{2015}

\vspace{-0.3cm}
\entry
{\textbf{Invited}}
{\textnormal{To the Dutch Organization of Internet Providers (NBIP)}}
{2015}

\vspace{-0.3cm}
\entry
{\textbf{Invited}}
{\textnormal{To the Dutch National Cyber Security Centrum (NCSC)}}
{2015}

\vspace{-0.3cm}
\entry
{\textbf{Abstract}}
{\textnormal{36$^{th}$ IRTF Network Management Research Group (NMRG)}}
{2015}

\vspace{-0.3cm}
\entry
{\textbf{Abstract}}
{\textnormal{31$^{th}$ IRTF Network Management Research Group (NMRG)}}
{2013}


\end{entrylist}

%==============================================================================
%==============================================================================
%==============================================================================

\section{Rewards \& Awards}\vspace{-5pt}

\begin{entrylist}

\vspace{-0.3cm}
\entry
{2016}
{Best Poster Presentation}
{2nd Cyber Security Workshop in the Netherlands}

\vspace{-0.3cm}
\entry
{2016}
{ACM SIGCOMM Travel Grant}
{Internet Measurement Conference (IMC)}

\vspace{-0.3cm}
\entry
{2016}
{IFIP/DMTF Travel Grant}
{Conference on Network and Service Management (CNSM)}

\vspace{-0.3cm}
\entry
{2016}
{Invited to the Dagstuhl Seminar 16361}
{on Security Challenges and Opportunities of SDN}


\vspace{-0.3cm}
\entry
{2015}
{Best Paper Award}
{Autonomous Infrastructure, Management and Security Conference}

\vspace{-0.3cm}
\entry
{2015}
{Best Ph.D. 'Elevator Pitch' (1$^{st}$ Place)}
{CTIT Symposium, University of Twente}

\vspace{-0.3cm}
\entry
{2015}
{Best Poster Presentation (3$^{rd}$ Place)}
{CTIT Symposium, University of Twente}

\vspace{-0.3cm}
\entry
{2015}
{Invited to the Dagstuhl Seminar 16012}
{on Global Measurements}

%\vspace{-0.3cm}
%\entry
%{2015}
%{Top Papers}
%{Brazilian Symposium on Computer Networks and Distributed Systems}

\vspace{-0.3cm}
\entry
{2010}
{Student Award (3$^{rd}$ Place)}
{Computer Engineer, Federal University of Pará, Brazil}

\end{entrylist}

%==============================================================================
%==============================================================================
%==============================================================================
%\section{Capacity to Stablish Collaboration}\vspace{-5pt}
%\url{Booking.com}, \url{Nationale Beheersorganisatie Internet Providers(NBIP)},
%\url{fox-it.com}, \url{quarantainenet}, \url{SURFNet}, \url{CESNET}, \url{RIPE NCC}

%==============================================================================
%==============================================================================
%==============================================================================
%\newpage
%\section{Scientific Appendix}\vspace{-5pt}
%
%\subsection{Publications}
%\vspace{-1cm}
%\nocite{*}
%\printbibliography[title={\emptyset}]

\end{document}

