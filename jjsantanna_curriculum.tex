%\documentclass[print]{styles/friggeri-cv-mac} % Add 'print' as an option into the square bracket to remove colors from this template for printing
\documentclass[print]{styles/friggeri-cv-linux} % Add 'print' as an option into the square bracket to remove colors from this template for printing

\usepackage{soul}
\addbibresource{mylibrary.bib}


\begin{document}

\header{José Jair }{Santanna}{Enthusiastic, Teacher, Researcher, Network Security Specialist \& (Big) Data Analyst} 

\begin{aside} 
 \section{Contact}
University of Twente
DACS research group
Drienerlolaan 5 
7522 NB Enschede
The Netherlands
~
\href{mailto:j.j.santanna@utwente.nl}{j.j.santanna@utwente.nl}
+31 53 4892505
~
\section{Links}\hspace{-1cm}
\hspace{-0.5cm}\href{http://www.jairsantanna.com}{jairsantanna.com}\includegraphics[scale=0.4]{img/jairsantanna.png}
\href{https://scholar.google.com/citations?user=TxcQNxUAAAAJ}{/TxcQNxUAAAAJ}\includegraphics[scale=0.3]{img/googlescholar.png}
\href{http://orcid.org/0000-0002-8361-6729}{/0...2-8361-6729}\includegraphics[scale=0.3]{img/orcid.png}
\href{http://www.researcherid.com/rid/K-2169-2016}{/K-2169-2016}\includegraphics[scale=0.3]{img/researchid.png}
\href{http://www.slideshare.net/jjcsantanna}{/jjcsantanna}\includegraphics[scale=0.3]{img/slideshare.png}
\href{https://www.linkedin.com/in/jjcsantanna}{/jjcsantanna}\includegraphics[scale=0.3]{img/linkedin.png}
\href{https://github.com/jjsantanna}{/jjsantanna\includegraphics[scale=0.3]{img/github.png}}
%\href{https://www.facebook.com/jjcsantanna}{/jjcsantanna}\includegraphics[scale=0.3]{img/facebook.png}
%~
%\section{Programming}
%	Bash, Python, C\#,SQL
%	PIG, Spark/Scala \& Impala
~
\section{Languages}
Dutch\includegraphics[scale=0.40]{img/2stars.png}
Spanish\includegraphics[scale=0.40]{img/3stars.png}
English\includegraphics[scale=0.40]{img/5stars.png}
Portuguese\includegraphics[scale=0.40]{img/5stars.png}
\end{aside}

\section{Professional Objective}\vspace{-10pt}
\setlength\parindent{12pt}\textit{I am currently a PhD candidate at University of
Twente with a postdoctoral contract starting right after achieve the
Doctor degree (*march 2017). My research is founded by the European commission
and by the Dutch national Organisation for Scientific Research (NWO). Some
results of my research already achieved recognition in the national and
international level; for example, I had the opportunity to advise the
Japanese Ministry of Internal Affairs and Communications and the Dutch National
Cyber Security Center both on the topic of cyber security.} 

\vspace{-5pt}\setlength\parindent{12pt}\textit{My professional objective is to become a
recognized researcher and teacher in network management and security. On top
of that I am eager to make my name and the name of the academic institution that
I am part of as a worldwide reference in cyber security. I aim to transfer my
enthusiasm, knowledge, skills and experience into meaningful innovation to
improve the security in all types of networks, specially in the Internet.}

%==============================================================================
%==============================================================================
%==============================================================================
\setlength\parindent{0pt}
\section{Education}\vspace{-5pt}

\begin{entrylist}

\entry
{2013--mar/2017*}
{Doctor {\normalfont in Computer Science}}
{University of Twente, The Netherlands}
{\emph{``The DDoS as a \$ervice Phenomenon''.} \\
Supervisor: Prof. Dr. Ir. Aiko Pras\\	
\textit{This ongoing work has being used by both the security community and by
network operators. One of the outcomes that stands out is publicly available at
\href{http://booterblacklist.com}{`booterblacklist.com'} and was already
downloaded by more than four hundred users worldwide.}}

\entry
{2010--2012}
{Master {\normalfont in Computer Science}}
{Federal University of Rio Grande do Sul, Brazil}
{\emph{``A BPM-based Solution for Inter-domain Circuit Management''.}\\
Supervisor: Prof. Dr. Lisandro Zambenedetti Granville\\
\textit{The outcome of this work is still in production used by the Brazilian
National Education and Research Network and available at
\href{http://meican.cipo.rnp.br/}{`meican.cipo.rnp.br'}.}}

\entry
{2005--2010}
{Bachelor {\normalfont in Computer Engineer}}
{Federal University of Par\'a, Brazil}
{\emph{``A Comparative Evaluation of DCCP Protocol in Mesh Networks.''} \\ 
Supervisor: Dr. Kelvin Lopes Dias
}

\end{entrylist}

%==============================================================================
%==============================================================================
%==============================================================================
\section{Experience}\vspace{-5pt}
\subsection{Technical Adviser}\vspace{-5pt}
\begin{entrylist}
\entry
{2016}
{To the Yokohama National University}
{Japan}
{For the production of a report on ``Reflection \& Amplification DDoS attacks''
requested by the \emph{Japanese Ministry of Internal Affairs and
Communications}.}

\entry
{2016}
{To the Dutch National Cyber Security Center and the Scientific Research \& Documentation Center}
{The Netherlands}
{For the production of a report on ``Cyber metrics 2016: DDoS and Malware''
requested by the \emph{Dutch Ministry of Security and Justice}.} \end{entrylist}

\subsection{Assistant Professor}\vspace{-5pt}
\begin{entrylist}
\entry
{2012--2013}
{University of Santa Cruz do Sul (UNISC)}
{Santa Cruz do Sul, Brazil}
{Tasks \& Responsibilities: Supervise students and teach (integrally) the courses: \textit{Digital
Systems, Computer Networks, and Network Management}.} 
\end{entrylist}

%==============================================================================
%==============================================================================
%==============================================================================
\subsection{Guest Lecturer}\vspace{-5pt}

\textit{All my slides are available at
\href{http://www.slideshare.net/jjcsantanna}{slideshare.net/jjcsantanna}; some
interactive exercises are available at
\href{https://github.com/jjsantanna/lectures_hands_on}{github.com/jjsantanna/lectures\_hands\_on};
and, most important, there is some voluntary judgment and testimonies of students
available at
\href{http://bit.ly/presentation_evaluation}{bit.ly/presentation\_evaluation}.}

\begin{entrylist}
\entry
{2016}
{Network Management}
{University of Twente, The Netherlands}
{Lectures: (1) `` SNMPv1, v2, v3 and Beyond'' and (2) ``Network management based
	on WebServices''\\
Audience: master students of computer science, electrical engineer, and
telematics from the University of Twente. }

\entry
{2015 \& 2016}
{Product Design to Online Business}
{University of Twente, The Netherlands}
{Lectures: (1) ``Booters: the DDoS-as-a-Service phenomenon'' and (2) ``Product
Design to Online Business''\\ Audience: bachelor students of industrial
engineering and business information technology.}

\entry
{2015}
{Network Security}
{University of Twente, The Netherlands}
{Lecture: ``DDoS attacks into the Matrix''\\
Audience: master and bachelor students of computer science, electrical engineer,
and telematics from the University of Twente, 3TU cyber security, and members of
ICT labs. }

\entry
{2015}
{Cybercrime \& Cybersecurity (minor)}
{University of Twente, The Netherlands}
{Lecture: 
Audience: bachelor students from University of Twente, European Research Center
for Information (ERCIS), Westfälische Wilhelms - Univerität Münster, Universität
Innsbruck, and University of Leicester. } \end{entrylist}

%==============================================================================
%==============================================================================
%==============================================================================
\subsection{Student Supervisor}\vspace{-5pt}

\textit{Among 10 supervised students: 5 had their research published in
international academic conferences, 5 decided to follow towards the next
academic degree [$^*$],i.e., Ph.D. or M.Sc., 2 received the M.Sc. degree
\textit{cum laude} [$^+$], 1 won the best paper award in an international
conference [$^\#$], and 1 won the second place of a renowned Dutch national
prize (KIVI/Defensie \& Veiligheid) for her thesis [$!$].}

\begin{entrylist}
\vspace{-0.3cm}
\entry
{[\textbf{M.Sc. Degree}]}
{\textnormal{Justyna Chromik[$^*$,$^+$,$!$] , Wouter de Vries[$^*$,$^\#$], Joey de
Vries[$^+$], Mark Wierbosch, and Jarmo van Lenthe.}}
{}

\vspace{-0.3cm}
\entry
{[\textbf{B.Sc. Degree}]}
{\textnormal{Roeland Krak[$^*$], Max Kerkers[$^*$], Dirk Maan[$^*$], Jochum Börger, and
Guilherme Dressler.}} 
{}

\end{entrylist}

\vspace{-0.3cm}
\textit{Besides those students I also supervised: two groups of students from University of
Twente and University of Münster in the ``Cyber Attacks and Cyber Security''
(minor); a group of 5 students in a ``Design Project''; and 1 student in the
``Crossing borders program''.}

%[MINOR in Cyber Attacks and Cyber Security ] Karel Kroonen, Michiel Hamilton,
%Rick Clement, Peter Schroten, Yuri van Midden, Benedict Steinhoff, Kilian
%Müller, Stephanie Massa, Sören Schleibaum [DESIGN Project] Tom Velthausz, Henk
%Mulder, Thomas Reesink, Jordy Michorius,Dennis Eijkel; [Crossing Border
%Module]Julik Keijer

%==============================================================================
%==============================================================================
%==============================================================================
\subsection{Journal and Conference Reviewer}\vspace{-5pt}

\textit{Among more than 20 venues acting as a reviewer I highlight bellow only
the top well ranked conferences by the Committee on Network Operation and
Management (CNOM), part of the IEEE Communications Society. For each conference
I added the most recent ratio between the number of submitted papers divided by
published (acceptance ratio) and their current h-index. You can find my entire
list of journals and conference that I was reviewer at
`jairsantanna.com/reviews'}.

\begin{entrylist}
\vspace{-0.3cm}
\entry
{COMCOM} %(2016)
{Computer Communications}
{2.099 | 49}

\vspace{-0.3cm}
\entry
{CNSM} %(2013-- 2016)
{International Conference on Network and Service Management}
{17.6\% | 18}

\vspace{-0.3cm}
\entry
{NETWORKING} %(2014--2016)
{IFIP Networking}
{23.8\% | 22 }

\vspace{-0.3cm}
\entry
{IM} %(2015)
{IFIP/IEEE Integrated Management}
{27.2\% | 18}

\vspace{-0.3cm}
\entry
{NOMS} %(2015)
{IEEE/IFIP Network Operations and Management Symposium }
{28.9\% | 22}

\end{entrylist}

%%%%FROM JEMS
%AIMS 	(2015)
%APNOMS (2013,2014)
%ERRC 	(2012,2013,2015,2016)
%IM 	(2015)	
%NOMS 	(2014, 2016)
%SBRC 	(2012)
%SBrT 	(2011,2012)
%SBSeg 	(2012) 
%WSL 	(2010)

%%%%FROM EDAS
%CCSNA	(2016)
%CNSM	(2016,2015,2014,2013)
%Networking (2016,2015,2014)
%EUNICE	(2013)
%ISCC	(2011)

%==============================================================================
%==============================================================================
%==============================================================================
\subsection{Conference Organizer}\vspace{-5pt}

\begin{entrylist}
\entry
{2016} 
{Shadow Technical Program Committee (TPC)}
{CNSM}
{\textit{IFIP/IEEE International Conference on Network and Service Management is
the premier conference in the area of network and service management. A shadow
TPC is an experimental initiative for PhDs become better TPCs in the future. My
reviews have the same weight as the actual TPC reviews.}}

\entry
{2013-2016} 
{Technical Program Committee (TPC)}
{ERRC}
{\textit{Escola Regional de Redes de Computadores (pt\_br), a conference for
undergrad students from the south of Brazil. To motivate students to }}

\entry
{2012} 
{Local Organizing Committee}
{LatinCloud}
{\textit{IEEE Latin American Conference on Cloud Computing and Communications.}}

\entry
{2010}
{Local Organizing Committee}
{SBRC}
{\textit{The Brazilian Symposium on Computer Networks and Distributed Systems.}}

\end{entrylist}

%==============================================================================
%==============================================================================
%==============================================================================
%\subsection{Project Activities}\vspace{-5pt}
%Management of Hybrid Networks (HyMan)

%Desenvolvimento de Tecnologia para Redes Metropolitanas Sem Fio voltada para Serviços de Cidades Inteligentes, Descrição: O projeto tem como objetivo desenvolver uma solução integrada para os componentes de escalonamento e controle de admissão para dispositivos de 4 geração utilizando a tecnologia WiMAX.. , Situação: Em andamento; Natureza: Pesquisa. , Alunos envolvidos: Graduação: (2) . , Integrantes: José Jair Cardoso de Santanna - Integrante / Cristiano Bonato Both - Coordenador.

%2007 - 2009
%Arcabouço para Acesso ao Conhecimento à População da Região Amazônica, 
%Descrição: Abordamos aspectos de pesquisa nas áreas de infra-estrutura de comunicações, implementação e avaliação de interfaces amigáveis para acesso universal aos cidadãos, considerando as oportunidades de conectividade existentes e peculiaridades da região amazônica. O estudo e solução propostos neste projeto consideram hardware de comunicação de baixo custo, além de software livre e aberto para a composição de um sistema de acesso flexível e escalável.. , Situação: Concluído; Natureza: Pesquisa. , Alunos envolvidos: Graduação: (3) / Mestrado acadêmico: (7) / Doutorado: (2) . , Integrantes: José Jair Cardoso de Santanna - Integrante / Kelvin Lopes Dias - Coordenador., Financiador(es): Conselho Nacional de Desenvolvimento Científico e Tecnológico - Auxílio financeiro.


%==============================================================================
%==============================================================================
%==============================================================================
%\subsection{Software Developer}\vspace{-5pt}
%\begin{entrylist}
%
%\entry
%{2015--Now}
%{Distributed Denial of Service Attacks Database (DDoSDB)}
%{\\Link: \href{http://ddosdb.org}{http://www.ddosdb.org} }
%{An initiative to develop the largest open database with DDoS attacks.\\
%Languages: Python, Bash, Pig Latin \& Apache Impala
%}
%
%\entry
%{2010--2011}
%{{Management Environment of Inter-domain Circuits for Advanced Networks (MEICAN)}}
%{Link: \href{http://meican.cipo.rnp.br/}{http://meican.cipo.rnp.br/}}
%{Development of an Executable Network Management Policies Editor sponsored by the Brazilian National Education and Research Network (RNP).    \\
%Language: WS-BPEL
%}

%\entry
%{2009}
%{Support for Planning Sustainable Electrical Power Generation via Vegetable Oil.}
%{}
%{Desktop application part of a Ph.D. thesis [\href{https://sites.google.com/site/ceamazonufpa/TeseAnaRosaDuarte.pdf}{link}] sponsored by a subsidiary (Eletronorte) of the major Brazilian power utility (Eletrobrás).\\
%Language: C\#.Net \& mySQL
%}
%\end{entrylist}



%==============================================================================
%==============================================================================
%==============================================================================
\section{Presentations}\vspace{-5pt}

\textit{In the last 3 years I gave more than 30 presentations, which is almost
one per month. Most of the presentations are available at
\href{http://www.slideshare.net/jjcsantanna}{`slideshare.net/jjcsantanna'}.
Among invited presentations, presentations that required abstract and 
presentations approved after a very strict peer-reviewing process (conferences)
I highlight the following.}

\begin{entrylist}

\vspace{-0.3cm}
\entry
{\textbf{Invited}}
{\textnormal{To Logius (part of the Dutch Ministry of Interior and Kingdom Relations)}}
{2015}

\vspace{-0.3cm}
\entry
{\textbf{Invited}}
{\textnormal{To the Dutch Organization of Internet Providers (NBIP)}}
{2015}

\vspace{-0.3cm}
\entry
{\textbf{Invited}}
{\textnormal{To the Dutch National Cyber Security Centrum (NCSC)}}
{2015}

\vspace{-0.3cm}
\entry
{\textbf{Abstract}}
{\textnormal{36$^{th}$ IRTF Network Management Research Group (NMRG)}}
{2015}

\vspace{-0.3cm}
\entry
{\textbf{Abstract}}
{\textnormal{31$^{th}$ IRTF Network Management Research Group (NMRG)}}
{2013}


\end{entrylist}

%==============================================================================
%==============================================================================
%==============================================================================
\section{Rewards \& Awards}\vspace{-5pt}

\begin{entrylist}

\vspace{-0.3cm}
\entry
{2016}
{Invited to the Dagstuhl Seminar 16361}
{on Network Attack Detection and Defense}

\vspace{-0.3cm}
\entry
{2015}
{Invited to the Dagstuhl Seminar 16012}
{on Global Measurements}

\vspace{-0.3cm}
\entry
{2015}
{Best Ph.D. 'Elevator Pitch' (1$^{st}$ Place)}
{CTIT Symposium, University of Twente}

\vspace{-0.3cm}
\entry
{2015}
{Best Poster (3$^{rd}$ Place)}
{CTIT Symposium, University of Twente}

\vspace{-0.3cm}
\entry
{2015}
{Best Paper Award}
{Autonomous Infrastructure, Management and Security Conference}

\vspace{-0.3cm}
\entry
{2015}
{Top Papers}
{Brazilian Symposium on Computer Networks and Distributed Systems}

\vspace{-0.3cm}
\entry
{2010}
{Student Award (3$^{rd}$ Place)}
{Computer Engineer, Federal University of Pará, Brazil}

\end{entrylist}

%==============================================================================
%==============================================================================
%==============================================================================
%\section{Capacity to Stablish Collaboration}\vspace{-5pt}
%\url{Booking.com}, \url{Nationale Beheersorganisatie Internet Providers(NBIP)},
%\url{fox-it.com}, \url{quarantainenet}, \url{SURFNet}, \url{CESNET}, \url{RIPE NCC}

%==============================================================================
%==============================================================================
%==============================================================================
\newpage
\section{Scientific Appendix}\vspace{-5pt}

\subsection{Publications}
\vspace{-1cm}
\nocite{*}
\printbibliography[title={\emptyset}]

\end{document}
